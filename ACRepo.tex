\documentclass[dvipdfmx]{jsarticle}
\usepackage{amsmath,amssymb}
\usepackage[dvipdfmx]{graphicx}
\usepackage{siunitx}
\usepackage{float}
\usepackage{tikz}
\usepackage{circuitikz}

\begin{document}
\section{目的}

我々が日常的に電力として利用している電気の多くは交流である。
また回路理論を学ぶ上で、交流及び交流回路の基本的な性質の理解は必要不可欠である。
本実験では、以下の3点を理解することを目的とする。

\begin{itemize}
\item
  交流電圧と周波数の測定方法
\item
  交流電圧波形の解析による交流の諸性質
\item
  交流回路における受動素子の構造と働き
\end{itemize}

\section{原理}

\subsection{正弦波交流の表式}

正弦波交流電圧\(v\)や電流\(i\)は次のような数式で表される。

\begin{equation}
v = V_m\sin(\omega t + \theta_v)
i = I_m\sin(\omega t + \theta_i)
\end{equation}

ここでは、\(V_m\)は電圧の最大値、\(\theta_v\)は電圧の位相角、\(\theta_i\)は電流の
位相角、\(t\)は時間\(\omega\)は角周波数を表す。
また、角周波数\(\omega\)と周期\(T\)、周波数\(f\)の関係は次のような式で表される。

\begin{equation}
\omega = \frac{2}{\pi} = 2\pi f[rad/s]
\end{equation}

\subsubsection{瞬時値と実効値}

ある時刻の交流の大きさを瞬時値と言う。実効値は、瞬時値の2乗を1周期の間平均した
値の平方根として定義され、交流の大きさを表すときに使われる。
実効値の物理的な意味は、「交流電圧(電流)を抵抗に加えたときに消費する電力が、
その実効値と同じ大きさの直流電圧(電流)を加えたときの消費電力と等しくなる。」
ということである。 実効値は次の式で求められる。

\begin{equation}
V = \frac{V_m}{\sqrt{2}}, I = \frac{I_m}{\sqrt2}
\end{equation}

\subsection{各種計測器で測定可能な電気的諸量}

本実験で使用する計測器で測定できる電気量を表1に示す。

\subsection{リサジュー図形を用いた位相差の測定}

リサジュー図形とは互いに直行する単振動が平面上に描く軌跡である。互いに直行する単振動を

\begin{equation}
	x(t) = A_1\sin(\omega_1 t + \theta_1) 
	x(t) = A_2\sin(\omega_2 t + \theta_2) 
\end{equation}

とする時、点
\end{document}